\documentclass[
%	year=2016,
%	month=October,
%	number=1,
%	volume=2,
	]{ijsra}
\def\IJSRAidentifier{\currfilebase}


\def\shorttitle{The \texttt{ijsra}-class, Version \IJSRAversion\ -- \IJSRAversiondate}
\def\maintitle{The \texttt{ijsra}-class, Version \IJSRAversion\ -- \IJSRAversiondate}
\def\shortauthor{Lukas C. Bossert}
\def\authormail{lukas@digitales-altertum.de}
\def\affiliation{Humboldt-Universität zu Berlin | Excellence-Cluster Topoi}
\def\thanknote{The coding of this documentclass is done at \href{https://github.com/LukasCBossert/documentclass-ijsra}{https://github.com/LukasCBossert/documentclass-ijsra} }
%\def\keywordname{hello}
\def\keywords{documentclass, \LaTeX , IJSRA}
%--------------------------------------------------------------


\begin{filecontents}{\IJSRAidentifier.bib}
@Article{Bossert-ijsra,
  author    = {Lukas C. Bossert},
  title     = {›ijsra‹ -- bib\LaTeX-style which is used for the \emph{International Journal of Student Research in Archaeology}},
  subtitle  = {Version: 0.1},
  url       = {http://www.ctan.org/pkg/biblatex-ijsra},
  keywords  = {latex},
  note      = {https://github.com/LukasCBossert/biblatex-ijsra},
  owner     = {Lukas C. Bossert},
  timestamp = {2016-07-04},
}


\end{filecontents}

\begin{document}
\IJSRAopening

	{\Large\scshape
	\shortauthor}%
	\footnote\thanknote%
	\\[1em]
	\email\\
	\affiliation

\IJSRAmid

\begin{IJSRAabstract}%
This is a documentation of the class \texttt{ijsra} which is used for the
 \emph{International Journal of Student Research in Archaeology}.
\end{IJSRAabstract}

%\IJSRAseparator

\lettrine[nindent=0em,lines=3]{E}{very}  article\marginnote{starting} starts with a lettrine as the first letter.
This bigger letter functions as an eye catcher to make clear where the article starts.
It effects the whole word: The first letter reaches down to the third line, the other letters of the word are capitalised. 
In \cref{lettrine} you see the code how this is achieved.

 \begin{lstlisting}[label=lettrine,caption={Starting letter}]
\lettrine[nindent=0em,lines=3]{E}{very} article ... 
\end{lstlisting}
If you have a letter which is rather slanted you can define the slope. 
Let’s have a look at the letter ›A‹.
 \begin{lstlisting}[label=lettrine,caption={Starting letter ›A‹}]
\lettrine[slope=4pt,findent=-3pt,lines=3]{A}{rchaeologists}  …
\end{lstlisting}
And it changes of course if we start the article with a ›W‹. 
 \begin{lstlisting}[label=lettrine,caption={Starting letter ›W‹}]
\lettrine[slope=-4pt,nindent=-4pt,lines=3]{W}{hen} ...
\end{lstlisting}

Each article contains several information about the author, title, etc. 
This is done with some definitions. 
You have to fill in the information in the curly brackets.
\begin{lstlisting}[label=information,caption={Information about the article}]
\def\IJSRAidentifier{\currfilebase} %<---- don’t change this!
\def\shorttitle{} %<---- this is for the short title
\def\maintitle{} %<---- full title
\def\shortauthor{} %<---- full name of the author
\def\authormail{} %<--- email address name@email.com
\def\affiliation{} %<--- university or institution
\def\thanknote{} %<--- further information regarding the author
\def\keywords{} %<---- keywords describing the article
%\def\keywordname{} %<---- name of keywords in an other language
\end{lstlisting}

After the definitions there is the block regarding the bibliographical entries.
Those are written in the section 
\begin{lstlisting}[label=bibliography,caption={Bibliographical information}]
\begin{filecontents}{\IJSRAidentifier.bib} %<--- don’t change this

@Incollection{Orengo2015,
  author       = {Orengo, H.},
  title        = {Open Source GIS and Geospatial Software in Archaeology},
  subtitle     = {Towards Their Integration into Everyday Archaeological Practice},
  pages        = {64--82},
  editor       = {Wilson, A. T. and Edwards, B.},
  booktitle    = {Open Source Archaeology},
  booksubtitle = {Ethics and Practice},
  publisher    = {deGruynter Open},
  location     = {Warsaw},
  year         = {2015},
}

@Article{Pikirayi2015,
  author       = {Pikirayi, I.},
  title        = {The Future of Archaeology in Africa},
  journaltitle = {Antiquity},
  volume       = {89},
  pages        = {531--541},
  year         = {2015},
  issue        = {345},
}
.
.
.

\end{filecontents}
\end{lstlisting}

After that we have completed the preamble and get to main part of the document.
Here we define the layout of the headline.

\begin{lstlisting}[label=headline,caption={Headline layout}]
\IJSRAopening %<---- don’t change or forget this
	{\Large\scshape
	\shortauthor}%
	\footnote\thanknote% 
	\\[1em]
	\email\\
	\affiliation
\IJSRAmid %<---- don’t change or forget this
\end{lstlisting}
If there is no \texttt{thanknote} you only have to comment line no. 4 and there will be no footnote.

After that we come to the abstract. 
The abstract is set in the enviroment \texttt{IJSRAabstract}:
\begin{lstlisting}[label=abstract,caption={IJSRAabstract}]
\begin{IJSRAabstract}
Abstract
\end{IJSRAabstract}
\end{lstlisting}

Then you can copy/paste the text of the article and make further editing.
At the very end you have to insert 
\begin{lstlisting}[label=closing,caption={IJSRAclosing}]
\IJSRAclosing
\end{lstlisting}

\clearpage
Following there is a minimal template how the plain document should look like when you begin to edit.
\begin{lstlisting}[label=document,caption={Plain document}]
\documentclass{ijsra}
\def\IJSRAidentifier{\currfilebase}
\def\shorttitle{}
\def\maintitle{}
\def\shortauthor{}
\def\authormail{}
\def\affiliation{}
\def\thanknote{}
\def\keywords{}
%\def\keywordname{}
\begin{filecontents}{\IJSRAidentifier.bib}
Bibliography-files
\end{filecontents}

\begin{document}
\IJSRAopening
	{\Large\scshape
	\shortauthor}%
	\footnote\thanknote%
	\\[1em]
	\email\\
	\affiliation
\IJSRAmid

\begin{IJSRAabstract}
Abstract
\end{IJSRAabstract}

\lettrine[nindent=0em,lines=3]{M}{ain} text ... 


\IJSRAclosing
\end{document}
\end{lstlisting}
\clearpage
Let’s have a look how to do certain editing.
We start with the figures,\marginnote{figures}

\clearpage
How to do quotes\marginnote{quotes}

\begin{IJSRAquote}{me}
greatgreatgreatgreatgreatgrea
greatgreatgreatgreatgreatgrea
tgreatgreatgreatgreatgreatgrea
tgreatgreatgreatgreatgreatgreatgreatgreatgreatgreatgreatgreat
\end{IJSRAquote}

\begin{lstlisting}[label=quote,caption={Quote}]
\begin{IJSRAquote}{me}
greatgreatgreatgreatgreatgrea
greatgreatgreatgreatgreatgrea
tgreatgreatgreatgreatgreatgrea
tgreatgreatgreatgreatgreatgreatgreatgreatgreatgreatgreatgreat
\end{IJSRAquote}
\end{lstlisting}

\clearpage
How to cite \marginnote{citing}



\clearpage
How to cite \marginnote{citing}

\begin{lstlisting}[label=Bossert-ijsra,caption={{@}Article\{Bossert-ijsra,…\} }]
@Article{Bossert-ijsra,
  author    = {Lukas C. Bossert},
  title     = {›ijsra‹ -- bib\LaTeX-style which is used for the \emph{International Journal of Student Research in Archaeology}},
  subtitle  = {Version: 0.1},
  url       = {http://www.ctan.org/pkg/biblatex-ijsra},
  keywords  = {latex},
  note      = {https://github.com/LukasCBossert/biblatex-ijsra},
  owner     = {Lukas C. Bossert},
  timestamp = {2016-07-04},
}
\end{lstlisting}

\clearpage
\nocite{*}
\IJSRAclosing
\end{document}