\documentclass[
%	year=2016,
%	month=October,
%	number=1,
%	volume=2,
	]{ijsra}
\def\IJSRAidentifier{\currfilebase}
%--------Author’s names------------
\def\authorone{Lukas C. Bossert}
%-------Biographical information-------------
\def\bioone{The coding of this documentclass is done at \href{https://github.com/LukasCBossert/documentclass-ijsra}{https://github.com/LukasCBossert/documentclass-ijsra} 
\alertwarning{Please know that this documentation may be deprecated since all developments are written down it the wiki of the journal’s repository: \url{https://github.com/LukasCBossert/ijsra/wiki}.}}
%-------Title-------------
\def\maintitle{The \texttt{ijsra}-class, Version \IJSRAversion\ -- \IJSRAversiondate}
\def\shorttitle{\maintitle}
\def\abstract{This is a documentation of the class \texttt{ijsra} which is used for the
 \emph{International Journal of Student Research in Archaeology}.
 Read this documentation carefully.
 When you start editing do it according to the structure shown in \cref{document}.}
%------University/Institution--------------
\def\affilone{digitales-altertum|de}
%--------Email------------
\def\cmail{lukas@digitales-altertum.de}
%\def\keywordname{}
\def\keywords{documentclass, \LaTeX , IJSRA}
%--------------------------------------------------------------


\begin{filecontents}{\IJSRAidentifier.bib}
@Incollection{Orengo2015,
  author       = {Orengo, H.},
  title        = {Open Source GIS and Geospatial Software in Archaeology},
  subtitle     = {Towards Their Integration into Everyday Archaeological Practice},
  pages        = {64--82},
  editor       = {Wilson, A. T. and Edwards, B.},
  booktitle    = {Open Source Archaeology},
  booksubtitle = {Ethics and Practice},
  publisher    = {deGruyter Open},
  location     = {Warsaw},
  year         = {2015},
}

@Article{Pikirayi2015,
  author       = {Pikirayi, I.},
  title        = {The Future of Archaeology in Africa},
  journaltitle = {Antiquity},
  volume       = {89},
  pages        = {531--541},
  year         = {2015},
  issue        = {345},
}

@Article{Bossert-ijsra,
  author    = {Lukas C. Bossert},
  title     = {›ijsra‹ -- bib\LaTeX-style which is used for the \emph{International Journal of Student Research in Archaeology}},
  subtitle  = {Version: 0.1},
  url       = {http://www.ctan.org/pkg/biblatex-ijsra},
  keywords  = {latex},
  year ={2016},
  note      = {https://github.com/LukasCBossert/biblatex-ijsra},
  owner     = {Lukas C. Bossert},
  timestamp = {2016-07-04},
}

\end{filecontents}

\lstMakeShortInline[style=code]|
\IJSRAopening


%\IJSRAseparator

\lettrine{E}{very}  article\IJSRAsection{starting} starts with a lettrine as the first letter.
This bigger letter functions as an eye catcher to make clear where the article starts.
It effects the whole word: The first letter reaches down to the third line, the other letters of the word are capitalised. 
In \cref{lettrine} you see the code how this is achieved.

 \begin{lstlisting}[label=lettrine,caption={Starting letter}]
\lettrine{E}{very} article ... 
\end{lstlisting}
If you have a letter which is rather slanted you can define the slope. 
Let’s have a look at the letter ›A‹.
 \begin{lstlisting}[label=lettrine,caption={Starting letter ›A‹}]
\lettrine[slope=4pt,findent=-3pt]{A}{rchaeologists}  …
\end{lstlisting}
And it changes of course if we start the article with a ›W‹. 
 \begin{lstlisting}[label=lettrine,caption={Starting letter ›W‹}]
\lettrine[slope=-4pt,nindent=-4pt]{W}{hen} ...
\end{lstlisting}

\IJSRAseparator
The |tex|-file \IJSRAsection{document class}of each article starts with the definition of the  document class.
Herefor we use the destinctive class |ijsra| and load it with this line:
\begin{lstlisting}[label=documentclass,caption={First line of tex-file.}]
\documentclass[<options>]{ijsra}
\end{lstlisting}
For  |<options>| you can use e.g. |draft|: this will omitt the figures and gives you a white frame with the size of the figures instead. 
The advantage is faster compiling.

\IJSRAseparator
Each article contains several information about the author(s), title etc. 
The document class can cope easily with one to five authors and their affiliation. 
Above that manual work is needed.
The information about author, title, affiliation, biography etc are written into curly brackets.
\begin{lstlisting}[label=information,caption={Information about the article}]
\def\IJSRAidentifier{\currfilebase} %<---- do not change this!
%-------Title | Email | Keywords | Abstract-------------
\def\shorttitle{Jons short title}
\def\maintitle{Jons very long title about his paper}
\def\cmail{John@Doe.com}%<---- corresponding email-address
\def\keywords{Research, Archaeology, ...}
%\def\keywordname{}%<--- redefine the name ›Keywords‹ in needed language
\def\abstract{In his paper Jon is showing ...}
%--------Author  names------------
\def\authorone{Jon Doe}
%\def\authortwo{}%<---- comment/delete if there is no second author.
%\def\authorthree{}%<---- comment/delete if there is no third author.
%\def\authorfour{}%<---- comment/delete if there is no fourth author.
%\def\authorfive{}%<---- comment/delete if there is no fifth author.
%-------Biographical information-------------
\def\bioone{Jon Doe is doing his research about ...}
%\def\biotwo{}%<---- comment/delete if there is no second author.
%\def\biothree{}%<---- comment/delete if there is no third author.
%\def\biofour{}%<---- comment/delete if there is no fourth author.
%\def\biofive{}%<---- comment/delete if there is no fifth author.
%------University/Institution--------------
\def\affilone{Jon Doe’s university or institution}
%\def\affiltwo{}%<---- comment/delete if there is no second author.
%\def\affilthree{}<---- comment/delete if there is no third author.
%\def\affilfour{}<---- comment/delete if there is no fourth author.
%\def\affilfive{}<---- comment/delete if there is no fifth author.
%--------Mapping of authors to their affiliations------------
%% authorone:--> * <--- copy/paste that symbol
%% authortwo:--> † <--- copy/paste that symbol
%% authorthree:--> ‡ <--- copy/paste that symbol
%% authorfour: --> § <--- copy/paste that symbol
%% authorfive: --> ¶ <--- copy/paste that symbol
%-----------------------------------------------------------------
%\def\affiloneauthor{}%<---- paste the symbol of the authors into {}
%\def\affiltwoauthor{}%<---- paste the symbol of the authors into {}
%\def\affilthreeauthor{}%<---- paste the symbol of the authors into {}
%\def\affilfourauthor{}%<---- paste the symbol of the authors into {}
%\def\affilfiveauthor{}%<---- paste the symbol of the authors into {}
\end{lstlisting}
\IJSRAseparator
After the definitions there is the block regarding the bibliographical entries.
Those are written in the section 
\begin{lstlisting}[label=bibliography,caption={Bibliographical information}]
\begin{filecontents}{\IJSRAidentifier.bib} %<--- do not change this

@Incollection{Orengo2015,
  author       = {Orengo, H.},
  title        = {Open Source GIS and Geospatial Software in Archaeology},
  subtitle     = {Towards Their Integration into Everyday Archaeological Practice},
  pages        = {64--82},
  editor       = {Wilson, A. T. and Edwards, B.},
  booktitle    = {Open Source Archaeology},
  booksubtitle = {Ethics and Practice},
  publisher    = {deGruyter Open},
  location     = {Warsaw},
  year         = {2015},
}

@Article{Pikirayi2015,
  author       = {Pikirayi, I.},
  title        = {The Future of Archaeology in Africa},
  journaltitle = {Antiquity},
  volume       = {89},
  pages        = {531--541},
  year         = {2015},
  issue        = {345},
}

@Article{Bossert-ijsra,
  author    = {Lukas C. Bossert},
  title     = {›ijsra‹ -- bib\LaTeX-style which is used for the \emph{International Journal of Student Research in Archaeology}},
  subtitle  = {Version: 0.1},
  url       = {http://www.ctan.org/pkg/biblatex-ijsra},
  keywords  = {latex},
  note      = {https://github.com/LukasCBossert/biblatex-ijsra},
  owner     = {Lukas C. Bossert},
  timestamp = {2016-07-04},
}
.
.
.

\end{filecontents}
\end{lstlisting}

After that we have completed the preamble and get to main part of the document.
Here we define the layout of the headline.

\begin{lstlisting}[label=headline,caption={Headline layout}]
\IJSRAopening %<---- do not change or forget this
\end{lstlisting}

\IJSRAseparator
Then you can copy/paste the text of the article and make further editing.
At the very end you have to insert 
\begin{lstlisting}[label=closing,caption={IJSRAclosing}]
\IJSRAclosing
\end{lstlisting}


Following there is a minimal template how the plain document should look like when you begin to edit and there is only one author.
\begin{lstlisting}[label=document,caption={Plain document}]
\documentclass{ijsra}
\def\IJSRAidentifier{\currfilebase} %<---- do not change this!
\def\shorttitle{Jons short title}
\def\maintitle{Jons very long title about his paper}
\def\cmail{John@Doe.com}
\def\keywords{Research, Archaeology, ...}
%\def\keywordname{}
\def\abstract{In his paper Jon is showing ...}
\def\authorone{Jon Doe}
\def\bioone{Jon Doe is doing his research about ...}
\def\affilone{Jon Doe’s university or institution}
\begin{filecontents}{\IJSRAidentifier.bib}
Bibliography-files
\end{filecontents}

\begin{document}
\IJSRAopening

\lettrine{M}{ain} text ... 

\IJSRAclosing
\end{document}
\end{lstlisting}
Since we only have one author there is no need to specify the connection of |\authorone| to |\affilone| since this is done automatically.
As soon as you have more than one author you need to do the mapping by copying the symbols ( *  †  ‡  §  ¶ ) into the right |\affiloneauthor| or |\affiltwoauthor|.

Before we have a closer look how to do certain editing
let me give you some advice about naming the |tex|-file and the related figures.
The |tex|-file has to be saved under an unique and individual name.
Usually this is the family name of the author: e.g. |winckelmann.tex|
If you think this is not enough because the author has quite a common family (as it might be with \emph{Johnson}) 
plese use a word from the title, e.g. |Johnson_prehistory.tex|.
Notice there must not be a blank in the name, use instead |_| or |-|.

The name of the |tex|-file is what we have defined as the |\IJSRAidentifier|, 
with that we can differentiate between all the texts from the authors.

We need this |IJSRAidentifier| again for the figures:
put the figures in a folder called |figures|.
Do not name the figures e.g. |figure1|, |figure2| etc. but use again this 
|IJSRAidentifier| (e.g. |Johnson_prehistory|) and name the figures like this:
|Johnson_prehistory_fig01.jpg| or |winckelmann_fig01.png|.
You can also name tables in the same way:
|Johnson_prehistory_tab01.jpg| or |winckelmann_tab01.png|.

\alertinfo{But please make sure that the \emph{IJSRAidentifier} is used with correct spelling of the author’s name etc.}

\IJSRAseparator
As \IJSRAsection{sections}a subdivison of the paper into section we use the code |\IJSRAsection{<<section name>>}|.

Be aware to insert the code \emph{after} the first word of the paragraph under the section heading. 

\begin{lstlisting}[label=section,caption={Section}]
This\IJSRAsection{New Paragraph} is a new paragraph under the section with the title ›New Paragraph‹.
\end{lstlisting}
There is also the possibility to do a \IJSRAsubsection{subsections}  |\IJSRAsubsection{<<subsection name>>}|. It will be displayed as a |\IJSRAsection{<<section name>>}| but in the pdf-bookmarks there is a differentiation.

\alertwarning{This code may conflict with wrapfigures since the section title and the wrapfigure use the margin space. 
In that case you have to move the wrapfigure further up or further down.}


\IJSRAseparator
How to do quotes:\IJSRAsection{quotes}
Quotes are inserted between the enviroment |IJSRAquote|.
Let me give you an example:

\begin{lstlisting}[label=quote,caption={Quote}]
\begin{IJSRAquote}{Johann Joachim Winckelmann}
Grace can never properly be said to exist without beauty; 
for it is only in the elegant proportions of beautiful forms 
that can be found that harmonious variety of line and motion 
which is the essence and charm of grace.
\end{IJSRAquote}
\end{lstlisting}
It will be displayed as this:
 
\begin{IJSRAquote}{Johann Joachim Winckelmann}
Grace can never properly be said to exist without beauty; 
for it is only in the elegant proportions of beautiful forms 
that can be found that harmonious variety of line and motion 
which is the essence and charm of grace.
\end{IJSRAquote}
Of course you can also use |\cite[page]{bibtex-key}| for the author of the quote.

\IJSRAseparator
You can\IJSRAsection{separator} insert a separation mark between some paragraphs,
as it is inserted automatically just before the bibliography. 
For that we edited a special form a separation mark. 
You can use it with |\IJSRAseparator|.
\IJSRAseparator
There  \IJSRAsection{figures}are two ways to insert a figure into the text.
First by using the plain enviroment |figure|

\begin{lstlisting}[label=figure,caption={figure}]
\begin{figure}
\includegraphics[width=\linewidth]{NAME_OF_THE_FIGURE_WITHOUT_SUFFIX}
\caption{<<Description of the figure>> 
        {\normalfont \\ \copyright\ by NAME OF COPYRIGHT HOLDER}}
\label{fig:NAME_OF_THE_FIGURE_WITHOUT_SUFFIX}
\end{figure}
\end{lstlisting}

Second by using the enhanced enviroment |wrapfigure| 

\begin{lstlisting}[label=wrapfigure,caption={wrapfigure}]
\begin{wrapfigure}{O}{0.5\textwidth} 
\centering
\includegraphics[width=\linewidth]{NAME_OF_THE_FIGURE_WITHOUT_SUFFIX}
\caption{<<Description of the figure>> 
        {\normalfont \\ \copyright\ by NAME OF COPYRIGHT HOLDER}}
\label{fig:NAME_OF_THE_FIGURE_WITHOUT_SUFFIX}
\end{wrapfigure} 
\end{lstlisting}
Since you (hopefully) place the figures into the folder named |figures| there is no need to tell the path, the figure name is sufficient. 
\IJSRAseparator
When suitable \IJSRAsection{caption}we try to insert figures as a |wrapefigure| into the text.
Otherwise we insert them into the |figure|-enviroment.

If you have a table which is saved as a figure, 
you have to make a change to the caption. 
I show you how to do that. 
\begin{lstlisting}[label=caption,caption={Modify the caption}]
\captionof{table}{Put the caption here.}
\end{lstlisting}

\IJSRAseparator
Modifying\IJSRAsection{references}  e.g. the caption is important for referencing in the text.
We use the special macro |\cref{label}| to reference to a figure, a table or anything else.
It makes it easier because we can omit to write ›fig.‹ or ›table‹ before the number of the figure or the table. 
The correct form (›fig.‹ etc.)  will be inserted automatically by \LaTeX .
If you like to have the first letter capitalised you can use |\Cref| instead.

Furthermore you can insert several |label| into a |\cref| by |\cref{label1,label2,label3}|.

\IJSRAseparator
If you\IJSRAsection{numbers \& units} have a number or a number with an unit, 
please use the macros which are provided my |\SI|.
Let me show you some examples.
\begin{labeling}{range phrase (1)}
\item[meter] |100 m| has to be written as |\SI{100}{\meter}|;
\item[percent] |75 %| has to be written as |\SI{75}{\percent}|;
\item[range phrase (1)] |10x10 m|  has to be written as |\SIrange{10}{10}{\meter}|;
\item[range phrase (2)] |10--15 cm|  has to be written as |\SIrange[range-phrase=--]{10}{15}{\centi\meter}|;
\item[number] |23.394| has to be written as |\num{23394}|;
\item[angle] |56°| has to be written as |\ang{56}|;
\item[round precision] If you want to make a round precision of your numbers you can set it yourself by |\SI[round-precision=2]{23.3390}{\meter}| which will make |23.34 m|.
\end{labeling}



\IJSRAseparator
We have\IJSRAsection{Different abstracts} the abstract of the article in english in the beginning,
just as shown in the template in \cref{document,abstract}.
If there is another abstract in a different language we put that at the end of the article.
For that we need also to redefine the keywords and usually the name \emph{keywords} according to the language of the abstract.

I give you an example how to edit a second abstract.
\begin{lstlisting}[label=abstract2,caption={Different abstract}]
\def\keywordname{Mots clés}
\def\keywords{keywords in French, \LaTeX , IJSRA, journal}
\foreignlanguage{french}{%
\begin{IJSRAabstract}
Abstract\IJSRAsection{Abstract (French)} in French: Copy and paste the abstract in the different language here!
\end{IJSRAabstract}
}
\end{lstlisting}
\def\keywordname{Mots clés}
\def\keywords{keywords in French, \LaTeX , IJSRA, journal}
\foreignlanguage{french}{%
\begin{IJSRAabstract}
Abstract\IJSRAsection{Abstract (French)} in French: Copy and paste the abstract in the different language here!
\end{IJSRAabstract}
}

%Do not forget to add the language in the options of the documentclass
%\begin{lstlisting}[label=abstract2,caption={Add the language of the abstract into the preamble}]
%\documentclass[
%	french,%<--- add the language if not English
%	]{ijsra}
%\end{lstlisting}


\IJSRAseparator
Global options: These options can be used and set to the current number:
|month|,
|year|,
|volume|,
|number|.

Furthermore you can set the document to |draft| or give the name of the title figure with |titlefigure|.
The title figure has to be within the folder |figures| and be in an aspect ratio of \SIrange{1000}{400}{pt}.

\IJSRAseparator
Some general information:
\begin{itemize}
\item Be aware of different brackets, e.g. |[] {}|;
These are part of \LaTeX -programming-language, every opening bracket needs a closing one.
\item Replace e.g. |19 %| with |\SI{19}{\percent}| otherwise everything after |%| will be omitted by \LaTeX;
\item Replace |&| with |\&|; 
\item Replace the citation of authors (e.g |Blesser \& Salter 2006|) with |\cite{Blesser2006}|;
	 if you have a page-range write: 	|\cite[23--45]{Blesser2006}|;
\item To compile with bibliography-references;
use 1 $\times$ \hologo{XeLaTeX}, then 1 $\times$ compiler ›biber‹, then 1 $\times$ \hologo{XeLaTeX}.
\item If you want to know how to cite properly
please have a look at the documentation of \texttt{biblatex-ijsra}.\footnote{\cite[see][]{Bossert-ijsra};\\ \url{http://mirrors.ctan.org//macros/latex/contrib/biblatex-contrib/biblatex-ijsra/biblatex-ijsra.pdf}}
\end{itemize}
\nocite{*}

\IJSRAclosing

%\clearpage
%You find the code using to write this documentation below.
%\lstinputlisting[language={[AlLaTeX]{TeX}}]{\jobname.tex}
